\pergunta{9. Considere que os processos da tabela a seguir estão aguardando para
serem executados e que cada um permanecerá na memória durante o tempo especificado. 
O sistema operacional ocupa uma área de 20KB no início da memória e gerencia a 
memória utilizando um algoritmo de particionamento dinâmico modificado. 
A memória total disponível no sistema é de 64KB e é alocada em blocos múltiplos 
de 4KB. Os processos são alocados de acordo com sua identificação (em ordem crescente) 
e irão aguardar até obter a memória que necessitam. Calcule a perda de
memória por fragmentação interna e externa sempre que um processo é colocado ou 
retirado da memória. O sistema operacional compacta a memória apenas quando existem 
duas ou mais partições livres adjacentes.}

\begin{center}
	\begin{tabular}{| c | c | c |} \hline
	Processo & Memória & Tempo \\ \hline
	1 & 30 KB & 5 \\ \hline
	2 & 6 KB & 10 \\ \hline
	3 & 36 KB & 5 \\ \hline
	\end{tabular}
\end{center}

O processo 1 é iniciado, sendo alocados 30 KB para ele na memória principal.
Juntamente com o processo 1, o processo 2 é alocado logo no primeiro endereço
de memória não alocado, após a memória destinada ao processo 1.   

O processo 3 deve esperar a finalização do processo 2, para que haja um bloco
de memória não alocado suficiente para ele.  Há memória não alocada suficiente
para o processo, mas não um bloco inteiro suficiente.  Este problema é chamado
de fragmentação externa \cite{ContiguousTechniques}.


