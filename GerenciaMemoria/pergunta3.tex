\pergunta{3. Suponha um sistema computacional com 64KB de memória principal e que utilize um
sistema operacional de 14KB que implemente alocação contígua de memória. Considere também
um programa de 90KB, formado por um módulo principal de 20KB e três módulos
independentes, cada um com 10KB,20KB e 30KB. Como o programa poderia ser executado
utilizando-se apenas a técnica de overlay?}
\\ \\
 90KB $\rightarrow$ programa;\\
 10, 20 e 30 KB módulos.\\ 
 64KB  $\rightarrow$ computador;\\
-14KB  $\rightarrow$ memória utilizada pelo sistema operacional;\\
------\\
 50KB  livres;\\
-20KB  $\rightarrow$ modulo principal do programa;\\
------\\
=30KB livres\\
-30KB overlay, já que o overlay possui o mesmo tamanho do maior módulo do programa.\\

Os módulos serão executados na área de overlay.\\
