\pergunta{10. Considerando as estratégias para escolha da partição dinamicamente, conceitue as estratégias
best-fit e worst-fit especificando prós e contras de cada uma.}

O alocador com estratégia \textsl{best-fit} procura o menor espaço de memória
não alocada, onde caiba o processo \cite{BowieMemoryAllocation}. Nesta
estratégia, pode haver uma lista ordenada por tamanho de blocos livres para aumentar
a eficiência da busca.

Na estratégia \textsl{worst-fit}, o gerenciador de memória coloca o processo
no maior bloco de memória não alocado.  A ideia nesta estratégia é que após
a alocação deste processo, irá sobrar a maior quantidade memória após o
processo, aumentando a possibilidade de, comparado ao \textsl{best-fit}, 
outro processo poder usar o espaço restante \cite{BowieMemoryAllocation}.
Assim, o \textsl{worst-fit} tende a causar menos fragmentações.

